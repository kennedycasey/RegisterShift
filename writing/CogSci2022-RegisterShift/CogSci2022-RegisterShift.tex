% Template for Cogsci submission with R Markdown

% Stuff changed from original Markdown PLOS Template
\documentclass[10pt, letterpaper]{article}

\usepackage{cogsci}
\usepackage{pslatex}
\usepackage{float}
\usepackage{caption}

% amsmath package, useful for mathematical formulas
\usepackage{amsmath}

% amssymb package, useful for mathematical symbols
\usepackage{amssymb}

% hyperref package, useful for hyperlinks
\usepackage{hyperref}

% graphicx package, useful for including eps and pdf graphics
% include graphics with the command \includegraphics
\usepackage{graphicx}

% Sweave(-like)
\usepackage{fancyvrb}
\DefineVerbatimEnvironment{Sinput}{Verbatim}{fontshape=sl}
\DefineVerbatimEnvironment{Soutput}{Verbatim}{}
\DefineVerbatimEnvironment{Scode}{Verbatim}{fontshape=sl}
\newenvironment{Schunk}{}{}
\DefineVerbatimEnvironment{Code}{Verbatim}{}
\DefineVerbatimEnvironment{CodeInput}{Verbatim}{fontshape=sl}
\DefineVerbatimEnvironment{CodeOutput}{Verbatim}{}
\newenvironment{CodeChunk}{}{}

% cite package, to clean up citations in the main text. Do not remove.
\usepackage{apacite}

% KM added 1/4/18 to allow control of blind submission
\cogscifinalcopy

\usepackage{color}

% Use doublespacing - comment out for single spacing
%\usepackage{setspace}
%\doublespacing


% % Text layout
% \topmargin 0.0cm
% \oddsidemargin 0.5cm
% \evensidemargin 0.5cm
% \textwidth 16cm
% \textheight 21cm

\title{Children's shift from CDS to ADS vocabulary across early
childhood}

\usepackage{booktabs}
\usepackage{longtable}
\usepackage{array}
\usepackage{multirow}
\usepackage{wrapfig}
\usepackage{float}
\usepackage{colortbl}
\usepackage{pdflscape}
\usepackage{tabu}
\usepackage{threeparttable}
\usepackage{threeparttablex}
\usepackage[normalem]{ulem}
\usepackage{makecell}
\usepackage{xcolor}

\author{{\large \bf Kennedy Casey (kbcasey@uchicago.edu)}  \AND {\large \bf Marisa Casillas (mcasillas@uchicago.edu)} \\Department of Comparative Human Development, University of Chicago \\ Chicago, IL 60637 USA}

\newlength{\cslhangindent}
\setlength{\cslhangindent}{1.5em}
\newenvironment{CSLReferences}%
  {}%
  {\par}

\begin{document}

\maketitle

\begin{abstract}
250-word abstract here

\textbf{Keywords:}
child-directed speech; word production; linguistic input; social
register; corpus analysis
\end{abstract}

\hypertarget{introduction}{%
\section{Introduction}\label{introduction}}

\hypertarget{methods}{%
\section{Methods}\label{methods}}

\hypertarget{corpus-and-item-selection}{%
\subsection{Corpus and item selection}\label{corpus-and-item-selection}}

We analyzed 8251 transcripts in the North American English collection of
the Child Language Data Exchange System (CHILDES) database (MacWhinney,
2000). The included transcripts were drawn from 52 individual corpora
and featured 980 children up to 7 years of age (range = 1--84 months,
\emph{M} = 33.5 months).

15 CDS/ADS word pairs were selected as test items based on their
appearance on the MacArthur-Bates Communicative Development Inventory
(Fenson et al., 1994) and their frequency of occurrence in CHILDES
(criterion: at least 100 child-produced tokens and 100 other-produced
tokens; see Table 1).

where the same object, animal, or routine could be reasonably labeled
with either form in typical communicative interactions with young
children.

\begin{table}[ht]
\centering
\fontsize{8}{10}\selectfont
\begin{tabular}{r>{}lrrrr}
  \toprule
\multicolumn{2}{c}{\textbf{ }} & \multicolumn{2}{c}{\textbf{CDS tokens}} & \multicolumn{2}{c}{\textbf{ADS tokens}} \\
\cmidrule(l{3pt}r{3pt}){3-4} \cmidrule(l{3pt}r{3pt}){5-6}
\textbf{} & \textbf{CDS/ADS word pair} & \textbf{Child} & \textbf{Other} & \textbf{Child} & \textbf{Other }\\ 
  \midrule
1 & \em{doggy/dog} & 2249 & 2644 & 3519 & 5113 \\ 
  2 & \em{kitty/cat} & 1552 & 3309 & 2779 & 4443 \\ 
  3 & \em{tummy/stomach} & 435 & 623 & 112 & 360 \\ 
  4 & \em{daddy/dad} & 9603 & 10048 & 2313 & 1031 \\ 
  5 & \em{mommy/mom} & 20294 & 17070 & 7616 & 2552 \\ 
  6 & \em{bunny/rabbit} & 1237 & 2597 & 1060 & 1397 \\ 
  7 & \em{duckie/duck} & 307 & 647 & 1933 & 3003 \\ 
  8 & \em{blankie/blanket} & 174 & 224 & 825 & 874 \\ 
  9 & \em{froggy/frog} & 154 & 434 & 970 & 1846 \\ 
  10 & \em{potty/bathroom} & 511 & 786 & 161 & 270 \\ 
  11 & \em{night night/goodnight} & 149 & 153 & 102 & 446 \\ 
  12 & \em{dolly/doll} & 745 & 1054 & 674 & 2697 \\ 
  13 & \em{horsey/horse} & 1149 & 1034 & 1749 & 2575 \\ 
  14 & \em{piggy/pig} & 405 & 1212 & 1276 & 2139 \\ 
  15 & \em{birdie/bird} & 399 & 588 & 1879 & 3358 \\ 
   \bottomrule
\end{tabular}
\caption{CHILDES frequency for 15 CDS/ADS word pairs. Child-produced counts 
                             include tokens produced only by the target child. All other 
                             speakers' productions are included in the other-produced counts.} 
\end{table}

\hypertarget{results}{%
\section{Results}\label{results}}

\hypertarget{discussion}{%
\section{Discussion}\label{discussion}}

\hypertarget{acknowledgements}{%
\section{Acknowledgements}\label{acknowledgements}}

We are grateful to the members of the University of Chicago Chatter Lab
and Northwestern University Child Language Lab for valuable discussion
and feedback on this work.

\hypertarget{references}{%
\section{References}\label{references}}

\setlength{\parindent}{-0.1in} 
\setlength{\leftskip}{0.125in}

\noindent

\hypertarget{refs}{}
\begin{CSLReferences}{1}{0}
\leavevmode\hypertarget{ref-fenson1994variability}{}%
Fenson, L., Dale, P. S., Reznick, J. S., Bates, E., Thal, D. J.,
Pethick, S. J., \ldots{} Stiles, J. (1994). Variability in early
communicative development. \emph{Monographs of the Society for Research
in Child Development}, i--185.

\leavevmode\hypertarget{ref-macwhinney2000childes}{}%
MacWhinney, B. (2000). \emph{The CHILDES project: The database} (Vol.
2). Psychology Press.

\end{CSLReferences}

\bibliographystyle{apacite}


\end{document}
