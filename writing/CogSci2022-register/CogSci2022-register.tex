% Template for Cogsci submission with R Markdown

% Stuff changed from original Markdown PLOS Template
\documentclass[10pt, letterpaper]{article}

\usepackage{cogsci}
\usepackage{pslatex}
\usepackage{float}
\usepackage{caption}

% amsmath package, useful for mathematical formulas
\usepackage{amsmath}

% amssymb package, useful for mathematical symbols
\usepackage{amssymb}

% hyperref package, useful for hyperlinks
\usepackage{hyperref}

% graphicx package, useful for including eps and pdf graphics
% include graphics with the command \includegraphics
\usepackage{graphicx}

% Sweave(-like)
\usepackage{fancyvrb}
\DefineVerbatimEnvironment{Sinput}{Verbatim}{fontshape=sl}
\DefineVerbatimEnvironment{Soutput}{Verbatim}{}
\DefineVerbatimEnvironment{Scode}{Verbatim}{fontshape=sl}
\newenvironment{Schunk}{}{}
\DefineVerbatimEnvironment{Code}{Verbatim}{}
\DefineVerbatimEnvironment{CodeInput}{Verbatim}{fontshape=sl}
\DefineVerbatimEnvironment{CodeOutput}{Verbatim}{}
\newenvironment{CodeChunk}{}{}

% cite package, to clean up citations in the main text. Do not remove.
\usepackage{apacite}

% KM added 1/4/18 to allow control of blind submission
\cogscifinalcopy

\usepackage{color}

% Use doublespacing - comment out for single spacing
%\usepackage{setspace}
%\doublespacing


% % Text layout
% \topmargin 0.0cm
% \oddsidemargin 0.5cm
% \evensidemargin 0.5cm
% \textwidth 16cm
% \textheight 21cm

\title{From \emph{doggy} to \emph{dog}: Developmental shifts in
children's use of register-specific words}

\usepackage{booktabs}
\usepackage{longtable}
\usepackage{array}
\usepackage{multirow}
\usepackage{wrapfig}
\usepackage{float}
\usepackage{colortbl}
\usepackage{pdflscape}
\usepackage{tabu}
\usepackage{threeparttable}
\usepackage{threeparttablex}
\usepackage[normalem]{ulem}
\usepackage{makecell}
\usepackage{xcolor}

\author{{\large \bf Kennedy Casey} \\ University of Chicago \\ \texttt{kbcasey@uchicago.edu} \And {\large \bf Marisa Casillas} \\ University of Chicago \\ \texttt{mcasillas@uchicago.edu}}

\newlength{\cslhangindent}
\setlength{\cslhangindent}{1.5em}
\newenvironment{CSLReferences}%
  {}%
  {\par}

\begin{document}

\maketitle

\begin{abstract}
Child-directed language (CDL) features words such as \emph{doggy},
\emph{night-night}, and \emph{tummy} that are rarely used in
adult-directed language (ADL). Characteristics of CDL variants, such as
diminutivization and reduplication, explain why they may be learned and
produced earlier by children. However, it is not yet clear how or when
children switch to using ADL equivalents---\emph{dog}, \emph{goodnight},
\emph{stomach}. Through analysis of speech transcripts from CHILDES and
the Language Development Project corpus, we show that children
significantly increase their production of ADL variants across age, with
the average CDL-to-ADL transition point at 2.5 years. Many of the
linguistic features that distinguish CDL vs.~ADL registers (e.g.,
lexical and syntactic complexity) similarly differentiated the local
speech contexts surrounding CDL vs.~ADL variants in children's input.
Notably, these differences emerged even in speech that was primarily
child-directed. Learners may therefore be able to capitalize on these
linguistic cues to support their discovery of register along with
context-appropriate CDL/ADL pair use.

\textbf{Keywords:}
child-directed language; word production; linguistic input; speech
register; corpus analysis
\end{abstract}

\hypertarget{introduction}{%
\section{Introduction}\label{introduction}}

Across their first few years of life, children come to know hundreds if
not thousands of words (Fenson et al., 1994; Mayor \& Plunkett, 2011).
Word production typically begins around age one, followed by a
vocabulary `explosion' or `spurt' during toddlerhood (Ganger \& Brent,
2004; see also McMurray, 2007), and continued, measurable increases in
vocabulary size thereafter (Rice \& Hoffman, 2015). Here, we investigate
one dimension of this dramatic developmental change: the appearance and
use of words from distinct registers.

Vocabulary first gets off the ground, in part, with words that are
specifically tailored to young learners (e.g., \emph{doggy} and
\emph{tummy}: Ferguson, 1964). Hallmark features of child-directed
language (CDL), such as iconicity (Laing, Vihman, \& Keren-Portnoy,
2017), diminutivization (Kempe, Brooks, \& Gillis, 2005), and
reduplication (Ota, Davies-Jenkins, \& Skarabela, 2018) have been shown
to support early word learning. These effects are in addition to the
cross-cutting influence of a word's frequency, concreteness, length, and
association with infancy (e.g., \emph{bottle} and \emph{bib}) on early
learnability (Braginsky et al., 2019; Frank et al., 2017; Perry et al.,
2018).

While CDL-specific words (e.g., \emph{doggy}, \emph{night-night},
\emph{tummy}) are overrepresented in children's early vocabularies, they
are eventually exchanged for ADL equivalents in most contexts
(\emph{dog}, \emph{goodnight}, \emph{stomach}). However, these words do
not fully disappear. Instead, they become designated for use in a
specific context---communication with infants and young children (e.g.,
Sachs \& Devin, 1976; Shatz \& Gelman, 1973).

The addition of a word like \emph{stomach} to a child's vocabulary may
mark their growing awareness that word choice should be tailored to the
current interactional context (e.g., Clark, 1997, 2018). We do not, at
present, know when children begin to make shifts from CDL to ADL word
use or precisely how such a shift may be supported or initiated. Our
investigation starts where this transition is most easily observed, with
CDL/ADL word pairs (e.g., \emph{doggy/dog},
\emph{night-night/goodnight}, \emph{tummy/stomach}), as opposed to words
that become less relevant with time (e.g., \emph{diaper} and
\emph{peekaboo}).

Some might expect the appearance of both CDL and ADL labels for the same
referent to be a problem for early word learning---particularly when the
variants have little overlap in phonological form (e.g.,
\emph{bunny/rabbit} vs.~\emph{doggy/dog}). Indeed, learners often assume
that new labels refer to new items (i.e., {``mutual exclusivity''}:
Markman \& Wachtel, 1988; see Lewis et al., 2020, for a recent
meta-analysis). Yet, children seem to learn multiple CDL/ADL variants
without issue (see Clark, 1990).

One potential way to explain children's learning of both labels is to
consider the social context of CDL vs.~ADL use. While labeling an animal
as \emph{doggy} vs.~\emph{dog} may not communicate anything distinct
about the referent itself, the production of one variant vs.~the other
may indicate something meaningful about \emph{who} is being addressed or
producing the label. That is, differences in register could serve to
`explain away' the otherwise problematic redundancy of multiple labels
in these pairs (Clark, 1990). Indirect evidence for this idea comes from
findings that the mutual exclusivity effect is modulated by children's
experience with multiple languages (Byers-Heinlein \& Werker, 2009;
Houston-Price, Caloghiris, \& Raviglione, 2010) as well as the social
conditions under which multiple labels are introduced (e.g., by speakers
of a familiar or unfamiliar race: Weatherhead et al., 2021). Further,
children's modifications to their own speech when talking to infants and
younger children (Sachs \& Devin, 1976; Shatz \& Gelman, 1973) and their
awareness of socially meaningful linguistic variation (Ikeda, Kobayashi,
\& Itakura, 2018; Liberman, Woodward, \& Kinzler, 2017; Soley \&
Sebastian-Galles, 2020) suggest that they may be able to recognize the
importance of social context for language use from relatively early in
development.

We hypothesize that children may contend with CDL/ADL word pairs by
associating the contrasting forms with different modes of use (i.e., by
classifying each variant as belonging to a distinct register). To test
this idea, we first need to establish (a) when children begin to shift
away from producing CDL-specific words, and (b) how children may be able
to use bottom-up linguistic input cues to associate lexical variants
with their associated registers (i.e., CDL vs.~ADL).

\hypertarget{current-investigation}{%
\subsection{Current investigation}\label{current-investigation}}

We examine a small but core subset of 15 CDL-specific words in English
(e.g., \emph{doggy}, \emph{night-night}, \emph{tummy}) that are
prevalent in children's early vocabularies but are eventually replaced
by ADL words (\emph{dog}, \emph{goodnight}, \emph{stomach}). In Study 1,
we analyze over 60,000 utterances of spontaneous speech from children up
to seven years of age to establish when ADL variants become more
dominant in children's own productions. That is, when do children switch
from primarily using CDL variants to primarily using ADL variants? Our
data suggest that the average age of CDL-to-ADL switch occurs around 2.5
years.

We then explore the features of children's input that could support this
switch by examining the extent to which CDL and ADL words are used in
distinct linguistic contexts. Further processing of nearly 70,000
non-target-child utterances (primarily from adult caregivers and
addressed to the target child) revealed that CDL and ADL variants
co-occur with reliably different patterns of prosodic, lexical, and
syntactic information---cues that likely help learners associate them
with different modes of use, or emerging representations of register.

Together, these studies push us to consider children's vocabulary
development not as a simple accumulation of words or numeric increase in
vocabulary size but rather a deepening and restructuring of the lexicon
with growing linguistic and social maturity. The words \emph{dog} and
\emph{stomach} do not entirely replace \emph{doggy} and
\emph{tummy}---rather, the contrasting forms become reserved for use
with different addressees.

\hypertarget{study-1-when-do-children-shift-from-cdl-to-adl-vocabulary}{%
\section{Study 1: When do children shift from CDL to ADL
vocabulary?}\label{study-1-when-do-children-shift-from-cdl-to-adl-vocabulary}}

We tracked children's use of 15 CDL/ADL word pairs (Table 1) from early
infancy up to age seven. Since CDL variants rarely appear in ADL, we
predicted that children would shift away from production of these
CDL-specific words with increasing age. That is, we expected to see
replacement of CDL variants with ADL variants in children's own speech
across time.

\hypertarget{method}{%
\section{Method}\label{method}}

\hypertarget{corpora}{%
\subsection{Corpora}\label{corpora}}

We analyzed 8,251 transcripts in the North American English collection
of the Child Language Data Exchange System (CHILDES) database
(MacWhinney, 2000) for children up to 7 years of age. The included
transcripts were drawn from 52 individual corpora and featured 980
children (age range = 1--84 months, \emph{M} = 33.5 months). To further
gain purchase on our question of interest with \emph{longitudinal} data,
we also analyzed children's productions in the Language Development
Project (LDP) corpus (see Huttenlocher et al., 2010; Rowe, 2008, for
further details regarding participating families, recording procedures,
and transcription). LDP data included 622 transcripts from 59
English-learning children recorded every 4 months for approximately 90
minutes from 14 to 58 months.

\begin{table}[!h]

\caption{\label{tab:tab:tab1}CHILDES frequency for 15 CDL/ADL word pairs. Child-produced counts include tokens produced only by the target child.}
\centering
\fontsize{7}{9}\selectfont
\begin{tabu} to \linewidth {>{}l>{\centering}X>{\centering}X>{\centering}X>{\centering}X}
\toprule
\multicolumn{1}{c}{\textbf{ }} & \multicolumn{2}{c}{\textbf{CDL tokens}} & \multicolumn{2}{c}{\textbf{ADL tokens}} \\
\cmidrule(l{3pt}r{3pt}){2-3} \cmidrule(l{3pt}r{3pt}){4-5}
\textbf{Pair} & \textbf{Child} & \textbf{Other} & \textbf{Child} & \textbf{Other}\\
\midrule
\em{doggy/dog} & 2,249 & 2,644 & 3,519 & 5,113\\
\em{kitty/cat} & 1,552 & 3,309 & 2,779 & 4,443\\
\em{tummy/stomach} & 435 & 623 & 112 & 360\\
\em{daddy/dad} & 9,603 & 10,048 & 2,313 & 1,031\\
\em{mommy/mom} & 20,294 & 17,070 & 7,616 & 2,552\\
\em{bunny/rabbit} & 1,237 & 2,597 & 1,060 & 1,397\\
\em{duckie/duck} & 307 & 647 & 1,933 & 3,003\\
\em{blankie/blanket} & 174 & 224 & 825 & 874\\
\em{froggy/frog} & 154 & 434 & 970 & 1,846\\
\em{potty/bathroom} & 511 & 786 & 161 & 270\\
\em{night night/goodnight} & 149 & 153 & 102 & 446\\
\em{dolly/doll} & 745 & 1,054 & 674 & 2,697\\
\em{horsey/horse} & 1,149 & 1,034 & 1,749 & 2,575\\
\em{piggy/pig} & 405 & 1,212 & 1,276 & 2,139\\
\em{birdie/bird} & 399 & 588 & 1,879 & 3,358\\
\bottomrule
\end{tabu}
\end{table}

\hypertarget{target-words}{%
\subsection{Target words}\label{target-words}}

Fifteen CDL/ADL word pairs (30 total target words) were selected based
on two criteria: the appearance of at least one variant on the
MacArthur-Bates Communicative Development Inventory (CDI, Fenson et al.,
1994), and sufficient frequency of occurrence in CHILDES (at least 100
child-produced tokens and 100 other-produced tokens per variant). Pairs
were also selected based on our own subjective judgment that the same
object, animal, routine, or body part could be reasonably labeled with
either variant by young children\footnote{While onomatopoeic words can
  be used in a similar manner to the CDL-specific words in our test set
  (e.g., \emph{choo-choo} serving as a CDL-specific label for
  \emph{train}, or \emph{quack-quack} for \emph{duck}), these iconic
  items were not included because they are primarily used as sound
  effects rather than labels for objects or animals (Skarabela, Pool, \&
  Ota, 2018). The polysemous nature of iconic word usage does not
  provide as clear of a test of replacement of CDL variants with ADL
  variants over time.}. Across all transcripts, 64,852 child-produced
utterances contained at least one target word and were included in our
analyses.

\hypertarget{results}{%
\section{Results}\label{results}}

We asked when CDL variants are replaced by ADL variants in children's
own speech. Using the \emph{lme4} package in R (Bates et al., 2015; R
Core Team, 2021), we fit a mixed-effects binomial logistic regression
model predicting children's production of CDL vs.~ADL variants, with
target child age (in months, scaled) as a single fixed effect. Random
slopes and intercepts for word pairs were also included\footnote{glmer(variant
  \(\sim\) age (months, scaled) + (1 + age \textbar{} word pair), family
  = binomial)}. For each target word token, variant was coded as either
0 (CDL) or 1 (ADL). Thus, the model captures, for each age, the
probability of using ADL variants over CDL variants.

Children significantly increased their production of ADL variants over
age (\(\beta\) = 0.54, \emph{SE} = 0.11, \emph{t} = 4.92, \emph{p}
\textless{} 0.001 (Figure 1). The average CDL-to-ADL transition point
(i.e., the point at which ADL variants were produced \textgreater50\% of
the time) was at approximately 28 months, or 2.5 years.

\begin{CodeChunk}
\begin{figure}[h]

{\centering \includegraphics{figs/shift-timing-fig-1} 

}

\caption[Model-predicted increase in production of ADL variants across age, with shaded standard error region]{Model-predicted increase in production of ADL variants across age, with shaded standard error region. Gray lines depict individual word-pair trajectories.}\label{fig:shift-timing-fig}
\end{figure}
\end{CodeChunk}

The trend of increasing ADL variant production was significant for 13 of
15 word pairs, but the exact trajectory of shift varied greatly across
pairs (Figure 2). In some cases, CDL variants were replaced by ADL
variants early on (e.g., \emph{doggy/dog} and \emph{kitty/cat} around 2
years). For other pairs, the CDL-to-ADL transition point was much later
(e.g., \emph{tummy/stomach} and \emph{potty/bathroom} around 5 years).
Finally, a clear transition point was not observed for some pairs
because ADL variants were produced \textgreater50\% of the time even at
the earliest ages sampled (e.g., \emph{duckie/duck} and
\emph{blankie/blanket}).

To further examine the robustness of the overall effect of increasing
ADL variant use over time, we ran subset analyses on all CHILDES
transcripts (primarily cross-sectional, with hundreds of children), all
LDP transcripts (longitudinal, \emph{n} = 59, age range = 14--58
months), and all transcripts from the Providence corpus, a small,
longitudinal subset of CHILDES (\emph{n} = 6, age range = 11--48 months:
Demuth, Culbertson, \& Alter, 2006). The main finding was replicated in
all three individual corpora. Children significantly increased their
production of ADL variants over age, collectively across all CHILDES
corpora (\(\beta\) = 0.55, \emph{SE} = 0.11, \emph{t} = 4.95, \emph{p}
\textless{} 0.001), as well as in the LDP corpus (\(\beta\) = 0.38,
\emph{SE} = 0.04, \emph{t} = 8.61, \emph{p} \textless{} 0.001) and
Providence corpus (\(\beta\) = 0.45, \emph{SE} = 0.14, \emph{t} = 3.23,
\emph{p} = 0.001). Moreover, the average CDL-to-ADL transition point was
estimated to be around 2.5 years in all corpora (CHILDES: 28 months,
LDP: 30 months, Providence: 27 months).

\begin{CodeChunk}
\begin{figure}[!ht]

{\centering \includegraphics{figs/shift-timing-bypair-fig-1} 

}

\caption[Individual word-pair trajectories for increasing production of ADL variants (blue) and decreasing production of CDL variants (red) with age]{Individual word-pair trajectories for increasing production of ADL variants (blue) and decreasing production of CDL variants (red) with age. Points indicate proportions for each 1-month age bin. Vertical gray lines at 28 months indicate the overall model-predicted CDL-to-ADL transition point across all words.}\label{fig:shift-timing-bypair-fig}
\end{figure}
\end{CodeChunk}

\hypertarget{discussion}{%
\section{Discussion}\label{discussion}}

Analysis of children's own spontaneous speech revealed developmental
shifts in their production of CDL vs.~ADL variants, with the latter
becoming increasingly more prominent over age. As an indicator of
robustness, this effect emerged in three different corpora with vastly
different sample sizes and distinct sampling strategies (i.e.,
cross-sectional vs.~longitudinal). We additionally found substantial
pair-level variation in the exact trajectories of CDL-to-ADL vocabulary
shift, but overall, we take children's shifts away from CDL variants and
toward ADL variants over time as indirect evidence of their early
formation of CDL and ADL as distinct registers.

\hypertarget{study-2-what-linguistic-information-in-childrens-input-supports-their-shift-from-cdl-to-adl-vocabulary}{%
\section{Study 2: What linguistic information in children's input
supports their shift from CDL to ADL
vocabulary?}\label{study-2-what-linguistic-information-in-childrens-input-supports-their-shift-from-cdl-to-adl-vocabulary}}

We next explored children's input (i.e., other-produced speech), asking
what linguistic information could support their shift from CDL to ADL
vocabulary. We conceptualize our second study as an investigation of the
cues that could help learners associate CDL and ADL variants with their
appropriate registers.

CDL, as a register, is differentiated from ADL at multiple linguistic
levels, including prosodic, lexical, and syntactic (e.g., Soderstrom,
2007). In English, CDL is associated with higher overall pitch as well
as greater variability in pitch contours (Fernald, 1989; Vosoughi \&
Roy, 2012). CDL utterances are often produced more slowly (e.g., Ko \&
Soderstrom, 2013; Vigliocco et al., 2020; but see Martin et al., 2016).
CDL typically includes less lexical diversity (Hills, 2013) and more
words that children already know (Foushee, Griffiths, \& Srinivasan,
2016). Syntactically, CDL is characterized as less complex than ADL. CDL
utterances are typically shorter (Brent \& Siskind, 2001; Martin et al.,
2016) and feature simpler constructions (Cameron-Faulkner, Lieven, \&
Tomasello, 2003).

Here, we tested whether the linguistic features that differentiate CDL
vs.~ADL at the register level also differentiate the local speech
contexts surrounding CDL vs.~ADL variants---even in speech that is
primarily addressed to children from their adult caregivers (i.e.,
language from a single register). In other words, can the appearance of
one variant vs.~the other be predicted on the basis of individual
utterance-level prosodic, lexical, or syntactic cues?

We hypothesized that utterances with CDL variants (vs.~ADL variants)
would be associated with (1) higher mean pitch, (2) greater pitch
variability, (3) slower speaking rates, and (4) less lexical complexity.
We also predicted that CDL utterances would contain (5) fewer rare
words, (6) fewer words overall, and (7) fewer verb phrases. If these
linguistic cues reliably differentiate CDL vs.~ADL word usage contexts,
then they could provide a viable source of information to support
children's ability to associate these lexical variants with their
corresponding registers.

\begin{CodeChunk}
\begin{figure*}[h]

{\centering \includegraphics{figs/ling-predictors-fig-1} 

}

\caption[Coefficient estimates for linguistic predictors of variant]{Coefficient estimates for linguistic predictors of variant. Positive main effects (left panel) indicate that utterances are more likely to contain ADL variants when they have higher values for that predictor (e.g., more verbs). Positive age interactions (right panel) indicate an increasing effect of the predictor with age. Error bars depict standard errors of the coefficient estimates, and filled circles represent significant effects (\textit{p} $<$ 0.05).}\label{fig:ling-predictors-fig}
\end{figure*}
\end{CodeChunk}

\hypertarget{method-1}{%
\section{Method}\label{method-1}}

\hypertarget{corpora-1}{%
\subsection{Corpora}\label{corpora-1}}

In addition to the child-produced utterances from Study 1, we analyzed
69,709 other-produced utterances (i.e., utterances not produced by the
target child) in the same CHILDES transcripts. The majority of
utterances were produced by children's primary caregivers (\emph{n} =
58,071, or 83.3\%). While, by and large, the utterances are not
annotated for addressee, our manual scanning of the CHILDES corpora
suggested that the vast majority of this speech is addressed to the
target child. Study 2 analyses exclude the LDP corpus because it has not
been comprehensively timestamped.

\hypertarget{linguistic-input-predictors}{%
\subsection{Linguistic input
predictors}\label{linguistic-input-predictors}}

All input analyses were conducted over individual utterances containing
at least one of the 30 target words from Study 1. We quantified
prosodic, lexical, and syntactic information to describe each utterance.

\hypertarget{prosodic-level}{%
\subsubsection{Prosodic level}\label{prosodic-level}}

We measured three types of prosodic information: \textbf{mean pitch}
(Hz), \textbf{pitch range} (Hz), and \textbf{speech rate} (words per
second). These measures were calculated over all timestamped utterances
in CHILDES (42.3\% of other-produced utterances, 41.4\% of
child-produced utterances). Utterances shorter than 58ms were excluded
from analysis\footnote{This lower bound was set by identifying the
  shortest possible duration of an utterance containing at least one
  word in four manually annotated North American English corpora (see
  Bergelson et al., 2019, for details).}. Pitch information was
extracted using Praat software (Boersma \& Weenink, 2016).

\hypertarget{lexical-level}{%
\subsubsection{Lexical level}\label{lexical-level}}

We measured two types of lexical information: complexity and rarity.
\textbf{Lexical complexity} was defined as the negative log proportion
of known words in each utterance (consistent with Foushee, Griffiths, \&
Srinivasan, 2016; Kidd, Piantadosi, \& Aslin, 2012). A word was
considered `known' if the age of acquisition (AoA) estimate (Fenson et
al., 1994; Frank et al., 2017) was less than or equal to the age of the
target child when they heard or produced the utterance. Utterances with
proportionally fewer known words are considered more lexically complex.
\textbf{Lexical rarity} was determined based on overall frequency in
CHILDES. For all words with at least 10 tokens\footnote{Manual checks
  revealed that many of the lowest-frequency words in CHILDES included
  idiosyncratic or erroneous transcriptions, so we excluded words with
  fewer than 10 tokens from our estimates of lexical rarity to reduce
  noise in this measure.}, we calculated a rarity score as the negative
log proportion of other-produced tokens in CHILDES (i.e., number of
tokens for a given word divided by the sum of all tokens of all words in
the full corpus). We then averaged the rarity scores for all individual
words in a given target utterance. Utterances with more low-frequency
words are considered more lexically rare.

\hypertarget{syntactic-level}{%
\subsubsection{Syntactic level}\label{syntactic-level}}

Syntactic measures included both the utterance \textbf{length} (in
words) and \textbf{number of verb phrases}. The number of words per
utterance was automatically extracted using the \emph{childesr} package
(Braginsky, Sanchez, \& Yurovsky, 2021). The number of verb phrases per
utterance was determined using \emph{spaCy3}, an automatic syntactic
parser (Honnibal et al., 2020).

\hypertarget{results-1}{%
\section{Results}\label{results-1}}

We ran individual mixed-effects binomial logistic regression models of
CDL vs.~ADL variant use for each of seven linguistic input predictors.
Models included fixed effects of linguistic predictor (scaled), target
child age (in months, scaled), and their interaction as well as random
intercepts for individual word pairs and speakers\footnote{glmer(variant
  \(\sim\) linguistic predictor (numeric, scaled) * age (months, scaled)
  + (1 \textbar{} word pair) + (1 \textbar{} speaker), family =
  binomial)}. For each target word token, the variant was coded as CDL
(0) or ADL (1), so coefficient estimates provide a measure of the
strength of association between a predictor and ADL variants.

First, we ran our models on all other-produced utterances (i.e.,
children's input). Main effects of linguistic predictors and
interactions with age are shown in Figure 3. All models also revealed a
positive main effect of target child age (all \emph{p}s \textless{}
0.001), confirming that adults, like children in Study 1, increase ADL
variant production as their child addressees get older.

At the prosodic level, we found significant effects for two of the three
input predictors tested. Utterance-level \textbf{pitch range} was not
predictive of variant (\(\beta\) = 0.003, \emph{SE} = 0.02, \emph{t} =
0.18, \emph{p} = 0.858) and did not significantly interact with age
(\(\beta\) = , \emph{SE} = , \emph{t} = , \emph{p} ). However,
utterance-level \textbf{mean pitch} was a negative predictor of ADL
variant (\(\beta\) = , \emph{SE} = , \emph{t} = , \emph{p} ). That is,
utterances with higher overall mean pitch were more likely to contain
CDL variants, with no significant interaction with age (\(\beta\) = ,
\emph{SE} = , \emph{t} = , \emph{p} ). \textbf{Speech rate} (i.e., words
produced per second) was a positive predictor of ADL variant (\(\beta\)
= , \emph{SE} = , \emph{t} = , \emph{p} ). Utterances spoken more
quickly were more likely to contain ADL variants. This input predictor
also negatively interacted with age (\(\beta\) = , \emph{SE} = ,
\emph{t} = , \emph{p} ), indicating a decreasing strength in predictive
power across developmental time.

At the lexical level, we found significant effects for both input
predictors. Utterances with higher levels of \textbf{lexical complexity}
(\(\beta\) = , \emph{SE} = , \emph{t} = , \emph{p} ) and \textbf{lexical
rarity} (\(\beta\) = , \emph{SE} = , \emph{t} = , \emph{p} ) were more
likely to contain ADL variants. Lexical complexity did not interact with
age (\(\beta\) = , \emph{SE} = , \emph{t} = , \emph{p} ); whereas,
lexical rarity negatively interacted with age such that there was a
decreasing effect of this predictor over time (\(\beta\) = , \emph{SE} =
, \emph{t} = , \emph{p} ).

At the syntactic level, we found significant effects of
\textbf{utterance length} and \textbf{number of verb phrases}.
Utterances with more words (\(\beta\) = , \emph{SE} = , \emph{t} = ,
\emph{p} ) and more verb phrases (\(\beta\) = , \emph{SE} = , \emph{t} =
, \emph{p} ) were more likely to contain ADL variants. Both linguistic
predictors negatively interacted with age (Length: \(\beta\) = ,
\emph{SE} = , \emph{t} = , \emph{p} ; Verbs: \(\beta\) = , \emph{SE} = ,
\emph{t} = , \emph{p} ), suggesting that the strength of these
predictors decreases across developmental time.

Across all levels of linguistic representation, children's productions
largely mirrored others' (all main effects and interactions shown in
Figure 3). That is, children's own utterances showed reliably different
patterns of prosodic, lexical, and syntactic information for utterances
with CDL vs.~ADL variants.

\hypertarget{discussion-1}{%
\section{Discussion}\label{discussion-1}}

Analyses of children's input revealed reliable differences in the
patterns of linguistic information surrounding CDL vs.~ADL variants.
Many of the prosodic, lexical, and syntactic features that broadly
differentiate CDL vs.~ADL registers similarly partitioned utterances
containing CDL vs.~ADL variants. Notably, these differences in local
speech context emerged even in language that was primarily addressed to
children from their primary caregivers (i.e., language likely from a
single register---CDL). This finding underscores the idea that register
production reflects stylistic linguistic choices by the person producing
the utterance and does not necessarily require the prototypical
communicative context (e.g., a caregiver can use ADL-like utterances
when talking to a young child).

While we do not yet know if these linguistic cues are indeed exploited
by learners, this study identifies which patterns appear learnable in
principle and which patterns are reflected in children's own
productions. More broadly, this work provides support for the
possibility that associations with CDL vs.~ADL registers help learners
grasp the differences in the contexts of CDL vs.~ADL variant use and
thereby support their gradual transition away from production of more
contextually-constrained CDL-specific words. A next step is to
experimentally test how well children across this age range perceive
words as CDL- or ADL-relevant given the surrounding linguistic context,
or how their expectations for hearing one variant vs.~the other may be
modulated by linguistic cues such as mean pitch, lexical complexity, and
utterance length.

\hypertarget{general-discussion}{%
\section{General Discussion}\label{general-discussion}}

In the current studies, we establish that children shift away from
production of CDL-specific words (e.g., \emph{doggy} and \emph{tummy})
over age. As predicted, these child-centric words are replaced by ADL
equivalents---\emph{dog} and \emph{stomach}---at least until they again
become relevant when talking to younger children. Further, we identify
patterns in children's linguistic input (i.e., other-produced speech)
that could support their discovery of associations between CDL/ADL
variants and their typical modes of use (i.e., incipient representations
of register).

\hypertarget{more-than-vocabulary-size-understanding-words-and-using-them-in-context}{%
\subsection{More than vocabulary size: Understanding words and using
them in
context}\label{more-than-vocabulary-size-understanding-words-and-using-them-in-context}}

By analyzing spontaneous language production in the present study, we
find variation in form that is often overlooked but may be crucial for
understanding how children's vocabularies develop. Widely-used
caregiver-reported (Fenson et al., 1994) and researcher-administered
(Dunn \& Dunn, 1965) vocabulary measures typically ask for a binary
indication of whether a child `knows' a word. For good reason, these
surveys and tests often gloss over variations in form. This
standardization helps with generalizing over many idiosyncrasies, which
allows for large-scale, even cross-linguistic, comparisons (e.g., Frank
et al., 2017, 2021). At the same time, glossing over this lexical
variation may present a missed opportunity to investigate more nuanced
but essential aspects of vocabulary development. The present findings on
the transition between CDL and ADL variants help demonstrate that
vocabulary development taps into other major aspects of children's
language learning, including their recognition of multiple levels of
linguistic information (e.g., prosodic, lexical, and syntactic) and
their broader socialization as individuals who can effectively deploy
language across variable contexts.

\hypertarget{developing-linguistic-and-social-knowledge-in-tandem}{%
\subsection{Developing linguistic and social knowledge in
tandem}\label{developing-linguistic-and-social-knowledge-in-tandem}}

Children's linguistic knowledge builds around and together with their
social knowledge. The lexical variants of CDL vs.~ADL registers are just
one example of socially meaningful linguistic variation in children's
input. Variation also appears across languages, dialects, accents, and
other types of registers (e.g., pedagogical, narrative, etc.). Over
time, children become increasingly aware of the fact that language style
is modulated by a variety of social factors, including the identities of
speakers (e.g., from different social groups: Liberman, Woodward, \&
Kinzler, 2017) along with their addressees (e.g., young children
vs.~adults: Ikeda, Kobayashi, \& Itakura, 2018; Soley \&
Sebastian-Galles, 2020). Children may therefore be able to leverage this
social knowledge when learning language. We see examples of this in the
context of word learning when children show flexibility in applying the
mutual exclusivity heuristic in accordance with the social conditions
under which new words are introduced (e.g., Weatherhead et al., 2021),
and here, too, children's shifts from CDL to ADL vocabulary are likely
supported by their emerging knowledge about the contexts in which
different registers are used.

We focus here on the issue of encountering multiple labels for the same
referent in early word learning, but children also face the inverse
problem---one label for many different referents (e.g., Casey et al.,
2021; Meylan et al., 2021). We see these puzzles of word learning as
interrelated---and given children's early success in contending with
both sources of variability, as evidence that learning happens at
multiple levels.

Rather than conceptualizing vocabulary development as the simple
tallying up of new `known' words in a relatively low-dimensional
semantic space, there is richness to be found in interactions with other
types of information (linguistic, social, etc.) and in analyses of
change over time. Exploring children's \emph{use} words in varying forms
and contexts can provide insight into the patterns of information that
supported their learning in the first place.

\vspace{1em}\fbox{\parbox[b][][c]{7.5cm}{\centering \textbf{Data Availability}\\
All anonymized data and analysis scripts can be found at the following link: \href{https://github.com/kennedycasey/RegisterShift}{https://github.com/kennedycasey/RegisterShift}.
}}

\hypertarget{acknowledgements}{%
\section{Acknowledgements}\label{acknowledgements}}

We are grateful to the members of the University of Chicago Chatter Lab
and Northwestern University Child Language Lab for valuable discussion
and feedback on this work. Research reported in this publication was
supported by the Eunice Kennedy Shriver National Institute of Child
Health \& Human Development of the National Institutes of Health under
Award Number P01HD040605; by a grant from the Successful Pathways from
School to Work initiative of the University of Chicago, funded by the
Hymen Milgrom Supporting Organization; by a grant from the Institute of
Education Sciences, U.S. Department of Education, through grant
R305A190467 to the University of Chicago; and by a grant from the
Spencer Foundation. The content is solely the responsibility of the
authors and does not necessarily represent the official views of the
National Institutes of Health, the Institute of Education Sciences, or
the U.S. Department of Education.

\hypertarget{references}{%
\section{References}\label{references}}

\setlength{\parindent}{-0.1in} 
\setlength{\leftskip}{0.125in}

\noindent

\hypertarget{refs}{}
\begin{CSLReferences}{1}{0}
\leavevmode\hypertarget{ref-bates2015fitting}{}%
Bates, D., Mächler, M., Bolker, B., \& Walker, S. (2015). Fitting linear
mixed-effects models using {lme4}. \emph{Journal of Statistical
Software}, \emph{67}(1), 1--48.

\leavevmode\hypertarget{ref-bergelson2019north}{}%
Bergelson, E., Casillas, M., Soderstrom, M., Seidl, A., Warlaumont, A.
S., \& Amatuni, A. (2019). What do {North American} babies hear? A
large-scale cross-corpus analysis. \emph{Developmental Science},
\emph{22}(1), e12724.

\leavevmode\hypertarget{ref-boersma2016praat}{}%
Boersma, P., \& Weenink, D. (2016). \emph{Praat software}.

\leavevmode\hypertarget{ref-braginsky2021childesr}{}%
Braginsky, M., Sanchez, A., \& Yurovsky, D. (2021). \emph{Childesr:
Accessing the 'CHILDES' database}.

\leavevmode\hypertarget{ref-braginsky2019consistency}{}%
Braginsky, M., Yurovsky, D., Marchman, V. A., \& Frank, M. C. (2019).
Consistency and variability in children's word learning across
languages. \emph{Open Mind}, \emph{3}, 52--67.

\leavevmode\hypertarget{ref-brent2001role}{}%
Brent, M. R., \& Siskind, J. M. (2001). The role of exposure to isolated
words in early vocabulary development. \emph{Cognition}, \emph{81}(2),
B33--B44.

\leavevmode\hypertarget{ref-byers2009monolingual}{}%
Byers-Heinlein, K., \& Werker, J. F. (2009). Monolingual, bilingual,
trilingual: Infants' language experience influences the development of a
word-learning heuristic. \emph{Developmental Science}, \emph{12}(5),
815--823.

\leavevmode\hypertarget{ref-cameron2003construction}{}%
Cameron-Faulkner, T., Lieven, E., \& Tomasello, M. (2003). A
construction based analysis of child directed speech. \emph{Cognitive
Science}, \emph{27}(6), 843--873.

\leavevmode\hypertarget{ref-caseyURmoving}{}%
Casey, K., Potter, C. E., Lew-Williams, C., \& Wojcik, E. H. (2021).
Moving beyond ``nouns in the lab": Using naturalistic data to understand
why infants' first words include uh-oh and hi. \emph{PsyArXiv}.

\leavevmode\hypertarget{ref-clark1990pragmatics}{}%
Clark, E. V. (1990). On the pragmatics of contrast. \emph{Journal of
Child Language}, \emph{17}(2), 417--431.

\leavevmode\hypertarget{ref-clark1997conceptual}{}%
Clark, E. V. (1997). Conceptual perspective and lexical choice in
acquisition. \emph{Cognition}, \emph{64}(1), 1--37.

\leavevmode\hypertarget{ref-clark2018conversation}{}%
Clark, E. V. (2018). Conversation and language acquisition: A pragmatic
approach. \emph{Language Learning and Development}, \emph{14}(3),
170--185.

\leavevmode\hypertarget{ref-demuth2006word}{}%
Demuth, K., Culbertson, J., \& Alter, J. (2006). Word-minimality,
epenthesis and coda licensing in the early acquisition of english.
\emph{Language and Speech}, \emph{49}(2), 137--173.

\leavevmode\hypertarget{ref-dunn1965peabody}{}%
Dunn, L. M., \& Dunn, L. M. (1965). \emph{Peabody picture vocabulary
test}.

\leavevmode\hypertarget{ref-fenson1994variability}{}%
Fenson, L., Dale, P. S., Reznick, J. S., Bates, E., Thal, D. J.,
Pethick, S. J., Tomasello, M., Mervis, C. B., \& Stiles, J. (1994).
Variability in early communicative development. \emph{Monographs of the
Society for Research in Child Development}, i--185.

\leavevmode\hypertarget{ref-ferguson1964baby}{}%
Ferguson, C. A. (1964). Baby talk in six languages. \emph{American
Anthropologist}, \emph{66}, 103--114.

\leavevmode\hypertarget{ref-fernald1989intonation}{}%
Fernald, A. (1989). Intonation and communicative intent in mothers'
speech to infants: Is the melody the message? \emph{Child Development},
1497--1510.

\leavevmode\hypertarget{ref-foushee2016lexical}{}%
Foushee, R., Griffiths, T., \& Srinivasan, M. (2016). Lexical complexity
of child-directed and overheard speech: Implications for learning.
\emph{Proceedings of the 38th Annual Conference of the Cognitive Science
Society}, 1697--1702.

\leavevmode\hypertarget{ref-frank2017wordbank}{}%
Frank, M. C., Braginsky, M., Yurovsky, D., \& Marchman, V. A. (2017).
Wordbank: An open repository for developmental vocabulary data.
\emph{Journal of Child Language}, \emph{44}(3), 677--694.

\leavevmode\hypertarget{ref-frank2021variability}{}%
Frank, M. C., Braginsky, M., Yurovsky, D., \& Marchman, V. A. (2021).
\emph{Variability and consistency in early language learning: The
wordbank project}. MIT Press.

\leavevmode\hypertarget{ref-ganger2004reexamining}{}%
Ganger, J., \& Brent, M. R. (2004). Reexamining the vocabulary spurt.
\emph{Developmental Psychology}, \emph{40}(4), 621.

\leavevmode\hypertarget{ref-hills2013company}{}%
Hills, T. (2013). The company that words keep: Comparing the statistical
structure of child-versus adult-directed language. \emph{Journal of
Child Language}, \emph{40}(3), 586--604.

\leavevmode\hypertarget{ref-honnibal2020spacy}{}%
Honnibal, M., Montani, I., Van Landeghem, S., \& Boyd, A. (2020).
\emph{{spaCy: Industrial-strength Natural Language Processing in
Python}}.

\leavevmode\hypertarget{ref-houston2010language}{}%
Houston-Price, C., Caloghiris, Z., \& Raviglione, E. (2010). Language
experience shapes the development of the mutual exclusivity bias.
\emph{Infancy}, \emph{15}(2), 125--150.

\leavevmode\hypertarget{ref-huttenlocher2010sources}{}%
Huttenlocher, J., Waterfall, H., Vasilyeva, M., Vevea, J., \& Hedges, L.
V. (2010). Sources of variability in children's language growth.
\emph{Cognitive Psychology}, \emph{61}(4), 343--365.

\leavevmode\hypertarget{ref-ikeda2018sensitivity}{}%
Ikeda, A., Kobayashi, T., \& Itakura, S. (2018). Sensitivity to
linguistic register in 20-month-olds: Understanding the
register-listener relationship and its abstract rules. \emph{Plos One},
\emph{13}(4), e0195214.

\leavevmode\hypertarget{ref-kempe2005diminutives}{}%
Kempe, V., Brooks, P. J., \& Gillis, S. (2005). Diminutives in
child-directed speech supplement metric with distributional word
segmentation cues. \emph{Psychonomic Bulletin \& Review}, \emph{12}(1),
145--151.

\leavevmode\hypertarget{ref-kidd2012goldilocks}{}%
Kidd, C., Piantadosi, S. T., \& Aslin, R. N. (2012). The goldilocks
effect: Human infants allocate attention to visual sequences that are
neither too simple nor too complex. \emph{PloS One}, \emph{7}(5),
e36399.

\leavevmode\hypertarget{ref-ko2013additive}{}%
Ko, E., \& Soderstrom, M. (2013). Additive effects of lengthening on the
utterance-final word in child-directed speech. \emph{Journal of Speech,
Language, and Hearing Research}, \emph{56}(1), 364--371.

\leavevmode\hypertarget{ref-laing2017salient}{}%
Laing, C. E., Vihman, M., \& Keren-Portnoy, T. (2017). How salient are
onomatopoeia in the early input? A prosodic analysis of infant-directed
speech. \emph{Journal of Child Language}, \emph{44}(5), 1117--1139.

\leavevmode\hypertarget{ref-lewis2020role}{}%
Lewis, M., Cristiano, V., Lake, B. M., Kwan, T., \& Frank, M. C. (2020).
The role of developmental change and linguistic experience in the mutual
exclusivity effect. \emph{Cognition}, \emph{198}, 104191.

\leavevmode\hypertarget{ref-liberman2017preverbal}{}%
Liberman, Z., Woodward, A. L., \& Kinzler, K. D. (2017). Preverbal
infants infer third-party social relationships based on language.
\emph{Cognitive Science}, \emph{41}, 622--634.

\leavevmode\hypertarget{ref-macwhinney2000childes}{}%
MacWhinney, B. (2000). \emph{The CHILDES project: The database} (Vol.
2). Psychology Press.

\leavevmode\hypertarget{ref-markman1988children}{}%
Markman, E. M., \& Wachtel, G. F. (1988). Children's use of mutual
exclusivity to constrain the meanings of words. \emph{Cognitive
Psychology}, \emph{20}(2), 121--157.

\leavevmode\hypertarget{ref-martin2016utterances}{}%
Martin, A., Igarashi, Y., Jincho, N., \& Mazuka, R. (2016). Utterances
in infant-directed speech are shorter, not slower. \emph{Cognition},
\emph{156}, 52--59.

\leavevmode\hypertarget{ref-mayor2011statistical}{}%
Mayor, J., \& Plunkett, K. (2011). A statistical estimate of infant and
toddler vocabulary size from CDI analysis. \emph{Developmental Science},
\emph{14}(4), 769--785.

\leavevmode\hypertarget{ref-mcmurray2007defusing}{}%
McMurray, B. (2007). Defusing the childhood vocabulary explosion.
\emph{Science}, \emph{317}(5838), 631--631.

\leavevmode\hypertarget{ref-meylan2021quantifying}{}%
Meylan, S., Mankewitz, J., Floyd, S., Rabagliati, H., \& Srinivasan, M.
(2021). Quantifying lexical ambiguity in speech to and from
english-learning children. \emph{PsyArXiv}.

\leavevmode\hypertarget{ref-ota2018choo}{}%
Ota, M., Davies-Jenkins, N., \& Skarabela, B. (2018). Why choo-choo is
better than train: The role of register-specific words in early
vocabulary growth. \emph{Cognitive Science}, \emph{42}(6), 1974--1999.

\leavevmode\hypertarget{ref-perry2018iconicity}{}%
Perry, L. K., Perlman, M., Winter, B., Massaro, D. W., \& Lupyan, G.
(2018). Iconicity in the speech of children and adults.
\emph{Developmental Science}, \emph{21}(3), e12572.

\leavevmode\hypertarget{ref-r2021}{}%
R Core Team. (2021). \emph{R: A language and environment for statistical
computing}. R Foundation for Statistical Computing.

\leavevmode\hypertarget{ref-rice2015predicting}{}%
Rice, M. L., \& Hoffman, L. (2015). Predicting vocabulary growth in
children with and without specific language impairment: A longitudinal
study from 2; 6 to 21 years of age. \emph{Journal of Speech, Language,
and Hearing Research}, \emph{58}(2), 345--359.

\leavevmode\hypertarget{ref-rowe2008child}{}%
Rowe, M. L. (2008). Child-directed speech: Relation to socioeconomic
status, knowledge of child development and child vocabulary skill.
\emph{Journal of Child Language}, \emph{35}(1), 185--205.

\leavevmode\hypertarget{ref-sachs1976young}{}%
Sachs, J., \& Devin, J. (1976). Young children's use of age-appropriate
speech styles in social interaction and role-playing. \emph{Journal of
Child Language}, \emph{3}(1), 81--98.

\leavevmode\hypertarget{ref-shatz1973development}{}%
Shatz, M., \& Gelman, R. (1973). The development of communication
skills: Modifications in the speech of young children as a function of
listener. \emph{Monographs of the Society for Research in Child
Development}, 1--38.

\leavevmode\hypertarget{ref-skarabela2018train}{}%
Skarabela, B., Pool, E., \& Ota, M. (2018). The train goes {`choo
choo'}: A corpus analysis of onomatopoeic words in child-directed speech
and early production. \emph{BUCLD 43}.

\leavevmode\hypertarget{ref-soderstrom2007beyond}{}%
Soderstrom, M. (2007). Beyond babytalk: Re-evaluating the nature and
content of speech input to preverbal infants. \emph{Developmental
Review}, \emph{27}(4), 501--532.

\leavevmode\hypertarget{ref-soley2020infants}{}%
Soley, G., \& Sebastian-Galles, N. (2020). Infants' expectations about
the recipients of infant-directed and adult-directed speech.
\emph{Cognition}, \emph{198}, 104214.

\leavevmode\hypertarget{ref-vigliocco2020child}{}%
Vigliocco, G., Shi, J., Gu, Y., \& Grzyb, B. (2020). Child directed
speech: Impact of variations in speaking-rate on word learning.
\emph{Proceedings of the 42nd Annual Meeting of the Cognitive Science
Society}, \emph{42}, 1043--1049.

\leavevmode\hypertarget{ref-vosoughi2012longitudinal}{}%
Vosoughi, S., \& Roy, D. K. (2012). A longitudinal study of prosodic
exaggeration in child-directed speech. \emph{Speech Prosody Special
Interest Group (SProSIG)}.

\leavevmode\hypertarget{ref-weatherhead2021putting}{}%
Weatherhead, D., Kandhadai, P., Hall, D. G., \& Werker, J. F. (2021).
Putting mutual exclusivity in context: Speaker race influences
monolingual and bilingual infants' word-learning assumptions.
\emph{Child Development}, \emph{92}(5), 1735--1751.

\end{CSLReferences}

\bibliographystyle{apacite}


\end{document}
